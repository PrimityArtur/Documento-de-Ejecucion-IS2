\renewcommand{\arraystretch}{1.3} % para mejor espaciado vertical
\begin{longtable}{|c|p{2cm}|p{9cm}|c|}
\caption{Nombre de la tabla} \\
\hline
\textbf{Código} & \textbf{Nombre} & \textbf{Descripción} & \textbf{Prioridad} \\
\hline
\endfirsthead

\multicolumn{4}{c}%
{{\bfseries \tablename\ \thetable{} -- continuación desde la página anterior}} \\
\hline
\textbf{Código} & \textbf{Nombre} & \textbf{Descripción} & \textbf{Prioridad} \\
\hline
\endhead

\hline \multicolumn{4}{r}{{Continúa en la siguiente página}} \\
\endfoot

\hline
\endlastfoot

%%%%%%%%%%%%%%%%%%%
RNF1 & Seguridad &
El sistema debe implementar medidas de seguridad que garanticen la confidencialidad e integridad de la información en procesos como registro, inicio de sesión, gestión de contraseñas, manejo de saldo y administración de usuarios. En particular, debe cumplir con lo siguiente:

\textbf{- Control de acceso a través de roles de usuario:} El sistema debe emplear un esquema de roles (administrador/cliente) que limite el acceso a funcionalidades o información según el tipo de usuario.

& Alta \\
\hline

%%%%%%%%%%%%%%%%%%%
RNF2 & Usabilidad &
El sistema debe proporcionar una experiencia de uso intuitiva, coherente y agradable para todos los usuarios (clientes y administradores). Para ello, debe cumplir con los siguientes criterios:

\textbf{- Diseño consistente:} El diseño debe ser homogéneo en todas las pantallas (botones, menús, tipografía, colores).

\textbf{- Manejo de errores:} Cuando se produzcan errores o excepciones, el sistema debe proporcionar mensajes que expliquen la causa de manera clara.
& Alta \\
\hline

%%%%%%%%%%%%%%%%%%%
\end{longtable}