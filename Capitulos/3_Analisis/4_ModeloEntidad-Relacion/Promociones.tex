% ==== Entidad PROMOCION ====
\renewcommand{\arraystretch}{1.3}
\begin{longtable}{|p{3.5cm}|p{10cm}|}
\caption{Diccionario de la entidad Promoción}
\label{tab:diccionarioPromocion} \\ 
\hline
\textbf{Nombre:} & Promoción \\
\hline
\textbf{Descripción:} & 
Campaña de descuento temporal que incentiva la compra de contenidos
en \textbf{QuickContentMedia}.  Contiene el porcentaje de descuento y
el rango de vigencia. \\ \hline
\endfirsthead

\multicolumn{2}{c}{\textbf{Continuación desde la página anterior}} \\ 
\endhead

\hline \multicolumn{2}{r}{{Continúa en la siguiente página}} \\ 
\endfoot

\hline
\endlastfoot

\multicolumn{2}{|p{13.5cm}|}{\textbf{ATRIBUTOS}} \\ \hline
\textbf{Atributo} & \textbf{Descripción} \\ \hline
descuento     & Porcentaje de descuento aplicado. \\ \hline
fecha\_inicio & Fecha de inicio de la promoción. \\ \hline
fecha\_fin    & Fecha de finalización de la promoción. \\ \hline

\multicolumn{2}{|p{13.5cm}|}{\textbf{RELACIONES}} \\ \hline
\textbf{Relación} & \textbf{Descripción} \\ \hline
aplica\,(1\,:\,N) & Una \textbf{Promoción} \textit{aplica} a muchos
                    \textbf{Contenidos}; cada contenido puede tener,
                    como máximo, una promoción activa. \\ \hline
\end{longtable}