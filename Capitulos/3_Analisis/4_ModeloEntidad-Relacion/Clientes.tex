% ==== Entidad CLIENTE ====
\renewcommand{\arraystretch}{1.3}
\begin{longtable}{|p{3.5cm}|p{10cm}|}
\caption{Diccionario de la entidad Cliente}
\label{tab:diccionarioCliente} \\ 
\hline
\textbf{Nombre:} & Cliente \\
\hline
\textbf{Descripción:} & 
Sub-tipo de \textbf{Usuario} que interactúa con la plataforma como
consumidor de contenidos.  Puede comprar, descargar, calificar,
recargar saldo y recibir regalos dentro de
\textbf{QuickContentMedia}. \\ \hline
\endfirsthead

\multicolumn{2}{c}{\textbf{Continuación desde la página anterior}} \\ 
\endhead

\hline \multicolumn{2}{r}{{Continúa en la siguiente página}} \\ 
\endfoot

\hline
\endlastfoot

\multicolumn{2}{|p{13.5cm}|}{\textbf{ATRIBUTOS}} \\ \hline
\textbf{Atributo} & \textbf{Descripción} \\ \hline
saldo       & Saldo disponible (entero no negativo) para compras o regalos. \\ \hline
excliente   & Indicador lógico que marca si la cuenta está activa o es un excliente. \\ \hline

\multicolumn{2}{|p{13.5cm}|}{\textbf{RELACIONES}} \\ \hline
\textbf{Relación} & \textbf{Descripción} \\ \hline
\textit{Especialización} & Un \textbf{Cliente} es una especialización de \textbf{Usuario}. \\ \hline
realiza(descarga)\,(N\,:\,1) & Cada cliente puede \emph{descargar} múltiples \textbf{Contenidos}. \\ \hline
realiza(calificacion)\,(N\,:\,1) & Un cliente puede \emph{calificar} varios contenidos. \\ \hline
emisor\,(1\,:\,N) & Como \textit{emisor}, un cliente puede enviar muchos \textbf{Regalos}. \\ \hline
receptor\,(1\,:\,N) & Como \textit{receptor}, un cliente puede recibir múltiples regalos. \\ \hline
tiene\_historial\,(1\,:\,1) & Cada cliente \emph{tiene} un registro en \textbf{Historial} que resume sus compras.\\ \hline
\end{longtable}