\subsection{Diccionario de entidades}

El diccionario de entidades está compuesto por el nombre de la entidad, su descripción, sus atributos y sus tipos. 

% ==== Entidad USUARIO ====
\renewcommand{\arraystretch}{1.3}
\begin{longtable}{|p{3.5cm}|p{10cm}|}
\caption{Diccionario de la entidad Usuario}
\label{tab:diccionarioUsuario} \\ \hline
\textbf{Nombre:} & Usuario \\ \hline
\textbf{Descripción:} & 
Entidad que representa a cualquier persona con credenciales para iniciar sesión en el portal \textbf{QuickContentMedia}.  
Sirve como super-tipo de \textbf{Cliente} y \textbf{Administrador}. \\ \hline
\endfirsthead

\multicolumn{2}{c}{\textbf{Continuación desde la página anterior}} \\ \hline
\endhead

\hline \multicolumn{2}{r}{{Continúa en la siguiente página}} \\ \hline
\endfoot

\hline
\endlastfoot

\multicolumn{2}{|p{13.5cm}|}{\textbf{ATRIBUTOS}} \\ \hline
\textbf{Atributo} & \textbf{Descripción} \\ \hline
username   & Identificador único de inicio de sesión.  
Tipo: \textit{string}. \\ \hline
contrasena & Contraseña cifrada asociada al usuario.  
Tipo: \textit{string}. \\ \hline
nombre     & Nombre de pila.  
Tipo: \textit{string}. \\ \hline
apellido   & Apellidos completos.  
Tipo: \textit{string}. \\ \hline
\end{longtable}

% ==== Entidad CLIENTE ====
\renewcommand{\arraystretch}{1.3}
\begin{longtable}{|p{3.5cm}|p{10cm}|}
\caption{Diccionario de la entidad Cliente}
\label{tab:diccionarioCliente} \\ \hline
\textbf{Nombre:} & Cliente \\ \hline
\textbf{Descripción:} & 
Sub-tipo de \textbf{Usuario} que interactúa como consumidor: compra, descarga, califica contenidos y participa en rankings. \\ \hline
\endfirsthead
\multicolumn{2}{c}{\textbf{Continuación desde la página anterior}} \\ \hline
\endhead
\hline \multicolumn{2}{r}{{Continúa en la siguiente página}} \\ \hline
\endfoot
\hline
\endlastfoot
\multicolumn{2}{|p{13.5cm}|}{\textbf{ATRIBUTOS}} \\ \hline
\textbf{Atributo} & \textbf{Descripción} \\ \hline
saldo      & Saldo disponible para compras.  
Tipo: \textit{Integer}. \\ \hline
excliente  & Indica si la cuenta se encuentra cerrada (pasa a ex-cliente).  
Tipo: \textit{boolean}. \\ \hline
\end{longtable}

% ==== Entidad ADMINISTRADOR ====
\renewcommand{\arraystretch}{1.3}
\begin{longtable}{|p{3.5cm}|p{10cm}|}
\caption{Diccionario de la entidad Administrador}
\label{tab:diccionarioAdministrador} \\ \hline
\textbf{Nombre:} & Administrador \\ \hline
\textbf{Descripción:} & 
Sub-tipo de \textbf{Usuario} con privilegios de gestión: alta/baja contenidos, creación de promociones, recarga de saldo y administración de categorías. \\ \hline
\endfirsthead
\multicolumn{2}{c}{\textbf{Continuación desde la página anterior}} \\ \hline
\endhead
\hline \multicolumn{2}{r}{{Continúa en la siguiente página}} \\ \hline
\endfoot
\hline
\endlastfoot
\multicolumn{2}{|p{13.5cm}|}{\textbf{ATRIBUTOS}} \\ \hline
\textbf{Atributo} & \textbf{Descripción} \\ \hline
acceso & Nivel o clave de acceso al módulo de administración.  
Tipo: \textit{boolean}. \\ \hline
\end{longtable}

% ==== Entidad CONTENIDO ====
\renewcommand{\arraystretch}{1.3}
\begin{longtable}{|p{3.5cm}|p{10cm}|}
\caption{Diccionario de la entidad Contenido}
\label{tab:diccionarioContenido} \\ \hline
\textbf{Nombre:} & Contenido \\ \hline
\textbf{Descripción:} & 
Archivo multimedia que el portal ofrece para descarga (imágenes, sonidos o videos) junto a sus metadatos de catalogación. \\ \hline
\endfirsthead
\multicolumn{2}{c}{\textbf{Continuación desde la página anterior}} \\ \hline
\endhead
\hline \multicolumn{2}{r}{{Continúa en la siguiente página}} \\ \hline
\endfoot
\hline
\endlastfoot
\multicolumn{2}{|p{13.5cm}|}{\textbf{ATRIBUTOS}} \\ \hline
\textbf{Atributo} & \textbf{Descripción} \\ \hline
nombre          & Título descriptivo del contenido.  
Tipo: \textit{string}. \\ \hline
autor           & Autor o creador del contenido.  
Tipo: \textit{string}. \\ \hline
descripcion     & Resumen o descripción ampliada.  
Tipo: \textit{string}. \\ \hline
precio          & Precio de venta en la moneda definida por el sistema.  
Tipo: \textit{integer}. \\ \hline
tamano\_archivo & Peso del archivo en bytes o MB.  
Tipo: \textit{double}. \\ \hline
archivo         & Datos binarios del fichero.  
Tipo: \textit{binary}. \\ \hline
\end{longtable}

% ==== Entidad TIPO_ARCHIVO ====
\renewcommand{\arraystretch}{1.3}
\begin{longtable}{|p{3.5cm}|p{10cm}|}
\caption{Diccionario de la entidad Tipo\_Archivo}
\label{tab:diccionarioTipoArchivo} \\ \hline
\textbf{Nombre:} & Tipo\_Archivo \\ \hline
\textbf{Descripción:} & 
Catálogo de extensiones y metainformación técnica de los contenidos. \\ \hline
\endfirsthead
\multicolumn{2}{c}{\textbf{Continuación desde la página anterior}} \\ \hline
\endhead
\hline \multicolumn{2}{r}{{Continúa en la siguiente página}} \\ \hline
\endfoot
\hline
\endlastfoot
\multicolumn{2}{|p{13.5cm}|}{\textbf{ATRIBUTOS}} \\ \hline
\textbf{Atributo} & \textbf{Descripción} \\ \hline
extension  & Extensión física (JPG, MP3, AVI, …).  
Tipo: \textit{string}. \\ \hline
mime\_type & Tipo MIME oficial (image/jpeg, audio/mpeg, …).  
Tipo: \textit{string}. \\ \hline
formato    & Categoría general (imagen, música, video).  
Tipo: \textit{string}. \\ \hline
\end{longtable}

% ==== Entidad CATEGORIA ====
\renewcommand{\arraystretch}{1.3}
\begin{longtable}{|p{3.5cm}|p{10cm}|}
\caption{Diccionario de la entidad Categoría}
\label{tab:diccionarioCategoria} \\ \hline
\textbf{Nombre:} & Categoría \\ \hline
\textbf{Descripción:} & 
Etiqueta dentro del árbol jerárquico que agrupa contenidos de temática similar.  Puede anidarse sin límite de profundidad. \\ \hline
\endfirsthead
\multicolumn{2}{c}{\textbf{Continuación desde la página anterior}} \\ \hline
\endhead
\hline \multicolumn{2}{r}{{Continúa en la siguiente página}} \\ \hline
\endfoot
\hline
\endlastfoot
\multicolumn{2}{|p{13.5cm}|}{\textbf{ATRIBUTOS}} \\ \hline
\textbf{Atributo} & \textbf{Descripción} \\ \hline
nombre & Nombre de la categoría.  
Tipo: \textit{string}. \\ \hline
\end{longtable}

% ==== Entidad PROMOCION ====
\renewcommand{\arraystretch}{1.3}
\begin{longtable}{|p{3.5cm}|p{10cm}|}
\caption{Diccionario de la entidad Promoción}
\label{tab:diccionarioPromocion} \\ \hline
\textbf{Nombre:} & Promoción \\ \hline
\textbf{Descripción:} & 
Descuento temporal aplicado a uno o varios contenidos. \\ \hline
\endfirsthead
\multicolumn{2}{c}{\textbf{Continuación desde la página anterior}} \\ \hline
\endhead
\hline \multicolumn{2}{r}{{Continúa en la siguiente página}} \\ \hline
\endfoot
\hline
\endlastfoot
\multicolumn{2}{|p{13.5cm}|}{\textbf{ATRIBUTOS}} \\ \hline
\textbf{Atributo} & \textbf{Descripción} \\ \hline
nombre       & Nombre identificador de la promoción.  
Tipo: \textit{string}. \\ \hline
descuento    & Porcentaje de descuento (0–100).  
Tipo: \textit{integer}. \\ \hline
fecha\_inicio & Fecha de inicio de la vigencia.  
Tipo: \textit{date}. \\ \hline
fecha\_fin   & Fecha de término de la vigencia.  
Tipo: \textit{date}. \\ \hline
\end{longtable}

% ==== Entidad REGALO ====
\renewcommand{\arraystretch}{1.3}
\begin{longtable}{|p{3.5cm}|p{10cm}|}
\caption{Diccionario de la entidad Regalo}
\label{tab:diccionarioRegalo} \\ \hline
\textbf{Nombre:} & Regalo \\ \hline
\textbf{Descripción:} & 
Representa el envío de uno o más contenidos de un cliente a otro como presente. \\ \hline
\endfirsthead
\multicolumn{2}{c}{\textbf{Continuación desde la página anterior}} \\ \hline
\endhead
\hline \multicolumn{2}{r}{{Continúa en la siguiente página}} \\ \hline
\endfoot
\hline
\endlastfoot
\multicolumn{2}{|p{13.5cm}|}{\textbf{ATRIBUTOS}} \\ \hline
\textbf{Atributo} & \textbf{Descripción} \\ \hline
abierto & Indica si el destinatario ya abrió el regalo.  
Tipo: \textit{boolean}. \\ \hline
\end{longtable}

% ==== Entidad CARRITO ====
\renewcommand{\arraystretch}{1.3}
\begin{longtable}{|p{3.5cm}|p{10cm}|}
\caption{Diccionario de la entidad Carrito}
\label{tab:diccionarioCarrito} \\ \hline
\textbf{Nombre:} & Carrito \\ \hline
\textbf{Descripción:} & 
Agrupa contenidos antes de realizar el pago. \\ \hline
\endfirsthead
\multicolumn{2}{c}{\textbf{Continuación desde la página anterior}} \\ \hline
\endhead
\hline \multicolumn{2}{r}{{Continúa en la siguiente página}} \\ \hline
\endfoot
\hline
\endlastfoot
\multicolumn{2}{|p{13.5cm}|}{\textbf{ATRIBUTOS}} \\ \hline
\textbf{Atributo} & \textbf{Descripción} \\ \hline
descuento\_aplicado & Porcentaje total de descuento obtenido por promociones vigentes.  
Tipo: \textit{integer}. \\ \hline
\end{longtable}

% ==== Entidad COMPRA ====
\renewcommand{\arraystretch}{1.3}
\begin{longtable}{|p{3.5cm}|p{10cm}|}
\caption{Diccionario de la entidad Compra}
\label{tab:diccionarioCompra} \\ \hline
\textbf{Nombre:} & Compra \\ \hline
\textbf{Descripción:} & 
Transacción en la que un cliente adquiere uno o más contenidos mediante pago. \\ \hline
\endfirsthead
\multicolumn{2}{c}{\textbf{Continuación desde la página anterior}} \\ \hline
\endhead
\hline \multicolumn{2}{r}{{Continúa en la siguiente página}} \\ \hline
\endfoot
\hline
\endlastfoot
\multicolumn{2}{|p{13.5cm}|}{\textbf{ATRIBUTOS}} \\ \hline
\textbf{Atributo} & \textbf{Descripción} \\ \hline
fecha\_y\_hora & Marca temporal exacta de la compra.  
Tipo: \textit{timestamp}. \\ \hline
\end{longtable}

% ==== Entidad DESCARGA ====
\renewcommand{\arraystretch}{1.3}
\begin{longtable}{|p{3.5cm}|p{10cm}|}
\caption{Diccionario de la entidad Descarga}
\label{tab:diccionarioDescarga} \\ \hline
\textbf{Nombre:} & Descarga \\ \hline
\textbf{Descripción:} & 
Registro histórico de cada vez que un cliente descarga un contenido que posee. \\ \hline
\endfirsthead
\multicolumn{2}{c}{\textbf{Continuación desde la página anterior}} \\ \hline
\endhead
\hline \multicolumn{2}{r}{{Continúa en la siguiente página}} \\ \hline
\endfoot
\hline
\endlastfoot
\multicolumn{2}{|p{13.5cm}|}{\textbf{ATRIBUTOS}} \\ \hline
\textbf{Atributo} & \textbf{Descripción} \\ \hline
fecha\_y\_hora & Fecha y hora en que se efectuó la descarga.  
Tipo: \textit{timestamp}. \\ \hline
\end{longtable}

% ==== Entidad VALORACION ====
\renewcommand{\arraystretch}{1.3}
\begin{longtable}{|p{3.5cm}|p{10cm}|}
\caption{Diccionario de la entidad Valoración}
\label{tab:diccionarioValoracion} \\ \hline
\textbf{Nombre:} & Valoración \\ \hline
\textbf{Descripción:} & 
Opinión numérica (1–10) que un cliente otorga a un contenido previamente descargado. \\ \hline
\endfirsthead
\multicolumn{2}{c}{\textbf{Continuación desde la página anterior}} \\ \hline
\endhead
\hline \multicolumn{2}{r}{{Continúa en la siguiente página}} \\ \hline
\endfoot
\hline
\endlastfoot
\multicolumn{2}{|p{13.5cm}|}{\textbf{ATRIBUTOS}} \\ \hline
\textbf{Atributo} & \textbf{Descripción} \\ \hline
puntuacion & Nota asignada (1–10).  
Tipo: \textit{integer}. \\ \hline
fecha      & Fecha en que se emitió la valoración.  
Tipo: \textit{timestamp}. \\ \hline
\end{longtable}

% ==== Entidad RANKING ====
\renewcommand{\arraystretch}{1.3}
\begin{longtable}{|p{3.5cm}|p{10cm}|}
\caption{Diccionario de la entidad Ranking}
\label{tab:diccionarioRanking} \\ \hline
\textbf{Nombre:} & Ranking \\ \hline
\textbf{Descripción:} & 
 Identifica una lista ordenada (por descargas o por puntuación) calculada periódicamente. \\ \hline
\endfirsthead
\multicolumn{2}{c}{\textbf{Continuación desde la página anterior}} \\ \hline
\endhead
\hline \multicolumn{2}{r}{{Continúa en la siguiente página}} \\ \hline
\endfoot
\hline
\endlastfoot
\multicolumn{2}{|p{13.5cm}|}{\textbf{ATRIBUTOS}} \\ \hline
\textbf{Atributo} & \textbf{Descripción} \\ \hline
fecha              & Fecha de generación del ranking.  
Tipo: \textit{date}. \\ \hline
tipo               & Criterio empleado (``descargas'' o ``puntuación'').  
Tipo: \textit{string}. \\ \hline
posicion\_actual   & Posición global del contenido dentro del top 10.  
Tipo: \textit{integer}. \\ \hline
posicion\_anterior & Posición de la semana anterior (si existía).  
Tipo: \textit{integer}. \\ \hline
\end{longtable}

\newpage
% ==== Entidad RANKING_CONTENIDO ====
\renewcommand{\arraystretch}{1.3}
\begin{longtable}{|p{3.5cm}|p{10cm}|}
\caption{Diccionario de la entidad Ranking\_Contenido}
\label{tab:diccionarioRankingContenido} \\ \hline
\textbf{Nombre:} & Ranking\_Contenido \\ \hline
\textbf{Descripción:} & 
Vinculan cada contenido con su posición en un \textbf{Ranking}.  No posee atributos adicionales. \\ \hline
\multicolumn{2}{|p{13.5cm}|}{\textbf{ATRIBUTOS}} \\ \hline
\multicolumn{2}{|p{13.5cm}|}{Sin atributos propios.} \\ \hline
\end{longtable}

% ==== Entidad RANKING_CLIENTE ====
\renewcommand{\arraystretch}{1.3}
\begin{longtable}{|p{3.5cm}|p{10cm}|}
\caption{Diccionario de la entidad Ranking\_Cliente}
\label{tab:diccionarioRankingCliente} \\ \hline
\textbf{Nombre:} & Ranking\_Cliente \\ \hline
\textbf{Descripción:} & 
 Vincula cada cliente con su posición en un ranking de actividad.  No posee atributos adicionales. \\ \hline
\multicolumn{2}{|p{13.5cm}|}{\textbf{ATRIBUTOS}} \\ \hline
\multicolumn{2}{|p{13.5cm}|}{Sin atributos propios.} \\ \hline
\end{longtable}