\subsection{Diccionario de relaciones}

El diccionario de relaciones está compuesto por el nombre de la relación, las entidades participantes, descripción y cardinalidad. 

% -----------------------------------------------------------
%  DICCIONARIO DE RELACIONES
%  (cada relación en una tabla independiente)
% -----------------------------------------------------------
\renewcommand{\arraystretch}{1.3}

% ==== Relación PERTENECE_A ====
\begin{longtable}{|p{3.5cm}|p{10cm}|}
\caption{Diccionario de la relación \texttt{pertenece\_a}}
\label{tab:rel_pertenece_a} \\ \hline
\textbf{Nombre:} & pertenece\_a \\ \hline
\textbf{Entidades:} & \textbf{CONTENIDO} (N) $\longleftrightarrow$ (1) \textbf{CATEGORIA} \\ \hline
\textbf{Descripción:} & Indica la categoría a la que se clasifica un contenido.  
Un contenido puede clasificarse en varias categorías, pero cada fila señala sólo una de ellas. \\ \hline
\textbf{Cardinalidad:} & N (Contenido) : 1 (Categoría) \\ \hline
\end{longtable}

% ==== Relación PERTENECE (jerarquía de categorías) ====
\begin{longtable}{|p{3.5cm}|p{10cm}|}
\caption{Diccionario de la relación \texttt{pertenece}}
\label{tab:rel_pertenece_cat} \\ \hline
\textbf{Nombre:} & pertenece \\ \hline
\textbf{Entidades:} & \textbf{CATEGORIA‐hija} (N) $\longleftrightarrow$ (1) \textbf{CATEGORIA‐padre} \\ \hline
\textbf{Descripción:} & Representa la jerarquía interna del catálogo (sub-categorías).  
Una categoría puede contener muchas sub-categorías, mientras que cada sub-categoría tiene una única categoría padre. \\ \hline
\textbf{Cardinalidad:} & N (Hija) : 1 (Padre) \\ \hline
\end{longtable}

% ==== Relación TIENE (Contenido – Tipo_Archivo) ====
\begin{longtable}{|p{3.5cm}|p{10cm}|}
\caption{Diccionario de la relación \texttt{tiene}}
\label{tab:rel_tiene_tipo} \\ \hline
\textbf{Nombre:} & tiene \\ \hline
\textbf{Entidades:} & \textbf{CONTENIDO} (N) $\longleftrightarrow$ (1) \textbf{TIPO\_ARCHIVO} \\ \hline
\textbf{Descripción:} & Asocia el tipo técnico (extensión, MIME) con cada archivo.  
Un mismo tipo de archivo puede corresponder a muchos contenidos. \\ \hline
\textbf{Cardinalidad:} & N (Contenido) : 1 (Tipo\_Archivo) \\ \hline
\end{longtable}

% ==== Relación APLICA (Promoción – Contenido) ====
\begin{longtable}{|p{3.5cm}|p{10cm}|}
\caption{Diccionario de la relación \texttt{aplica}}
\label{tab:rel_aplica} \\ \hline
\textbf{Nombre:} & aplica \\ \hline
\textbf{Entidades:} & \textbf{PROMOCION} (1) $\longleftrightarrow$ (N) \textbf{CONTENIDO} \\ \hline
\textbf{Descripción:} & Registra qué contenidos se ven afectados por una promoción temporal.  
Cada promoción se aplica a uno o varios contenidos; un contenido puede tener solo una promoción activa. \\ \hline
\textbf{Cardinalidad:} & 1 (Promoción) : N (Contenido) \\ \hline
\end{longtable}

% ==== Relación CONTIENE (Carrito – Contenido) ====
\begin{longtable}{|p{3.5cm}|p{10cm}|}
\caption{Diccionario de la relación \texttt{contiene}}
\label{tab:rel_contiene} \\ \hline
\textbf{Nombre:} & contiene \\ \hline
\textbf{Entidades:} & \textbf{CARRITO} (M) $\longleftrightarrow$ (N) \textbf{CONTENIDO} \\ \hline
\textbf{Descripción:} & Tabla puente que lista los artículos colocados en cada carrito de compra.  
Un carrito puede tener muchos contenidos y cualquier contenido puede aparecer en muchos carritos distintos. \\ \hline
\textbf{Cardinalidad:} & M (Carrito) : N (Contenido) \\ \hline
\end{longtable}

% ==== Relación CALIFICADO_EN (Contenido – Valoración) ====
\begin{longtable}{|p{3.5cm}|p{10cm}|}
\caption{Diccionario de la relación \texttt{calificado\_en}}
\label{tab:rel_calificado_en} \\ \hline
\textbf{Nombre:} & calificado\_en \\ \hline
\textbf{Entidades:} & \textbf{CONTENIDO} (1) $\longleftrightarrow$ (N) \textbf{VALORACION} \\ \hline
\textbf{Descripción:} & Une cada valoración con el contenido al que se refiere.  
Un contenido puede acumular múltiples valoraciones; cada valoración corresponde a un único contenido. \\ \hline
\textbf{Cardinalidad:} & 1 (Contenido) : N (Valoración) \\ \hline
\end{longtable}

\newpage
% ==== Relación CALIFICA (Cliente – Valoración) ====
\begin{longtable}{|p{3.5cm}|p{10cm}|}
\caption{Diccionario de la relación \texttt{califica}}
\label{tab:rel_califica} \\ \hline
\textbf{Nombre:} & califica \\ \hline
\textbf{Entidades:} & \textbf{CLIENTE} (1) $\longleftrightarrow$ (N) \textbf{VALORACION} \\ \hline
\textbf{Descripción:} & Indica qué cliente emitió cada valoración.  
Un cliente puede calificar muchos contenidos; cada valoración proviene de un único cliente. \\ \hline
\textbf{Cardinalidad:} & 1 (Cliente) : N (Valoración) \\ \hline
\end{longtable}

% ==== Relación DESCARGADO_EN (Contenido – Descarga) ====
\begin{longtable}{|p{3.5cm}|p{10cm}|}
\caption{Diccionario de la relación \texttt{descargado\_en}}
\label{tab:rel_descargado_en} \\ \hline
\textbf{Nombre:} & descargado\_en \\ \hline
\textbf{Entidades:} & \textbf{CONTENIDO} (1) $\longleftrightarrow$ (N) \textbf{DESCARGA} \\ \hline
\textbf{Descripción:} & Historial de descargas por contenido.  
Cada registro de descarga apunta a un contenido; un mismo contenido puede ser descargado muchas veces. \\ \hline
\textbf{Cardinalidad:} & 1 (Contenido) : N (Descarga) \\ \hline
\end{longtable}

% ==== Relación VENDIDO_EN (Compra – Contenido) ====
\begin{longtable}{|p{3.5cm}|p{10cm}|}
\caption{Diccionario de la relación \texttt{vendido\_en}}
\label{tab:rel_vendido_en} \\ \hline
\textbf{Nombre:} & vendido\_en \\ \hline
\textbf{Entidades:} & \textbf{COMPRA} (1) $\longleftrightarrow$ (N) \textbf{CONTENIDO} \\ \hline
\textbf{Descripción:} & Detalla qué contenidos forman parte de cada transacción de compra (líneas de factura).  
Una compra puede incluir un contenido y un contenido puede venderse en muchas compras distintas. \\ \hline
\textbf{Cardinalidad:} & N (Compra) : N (Contenido) \\ \hline
\end{longtable}

% ==== Relación REGALADO_EN (Regalo – Contenido) ====
\begin{longtable}{|p{3.5cm}|p{10cm}|}
\caption{Diccionario de la relación \texttt{regalado\_en}}
\label{tab:rel_regalado_en} \\ \hline
\textbf{Nombre:} & regalado\_en \\ \hline
\textbf{Entidades:} & \textbf{REGALO} (N) $\longleftrightarrow$ (1) \textbf{CONTENIDO} \\ \hline
\textbf{Descripción:} & Lista los contenidos incluidos dentro de cada regalo enviado entre clientes. \\ \hline
\textbf{Cardinalidad:} & N (Regalo) : 1 (Contenido) \\ \hline
\end{longtable}

\newpage
% ==== Relación ENVIA (Cliente – Regalo) ====
\begin{longtable}{|p{3.5cm}|p{10cm}|}
\caption{Diccionario de la relación \texttt{envia}}
\label{tab:rel_envia} \\ \hline
\textbf{Nombre:} & envia \\ \hline
\textbf{Entidades:} & \textbf{CLIENTE} (1) $\longleftrightarrow$ (N) \textbf{REGALO} \\ \hline
\textbf{Descripción:} & Registra al cliente que actúa como remitente de un regalo. \\ \hline
\textbf{Cardinalidad:} & 1 (Cliente) : N (Regalo) \\ \hline
\end{longtable}

% ==== Relación ENVIA/RECIBE (Cliente – Regalo) ====
\begin{longtable}{|p{3.5cm}|p{10cm}|}
\caption{Diccionario de la relación \texttt{envia/recibe}}
\label{tab:rel_envia_recibe} \\ \hline
\textbf{Nombre:} & envia/recibe \\ \hline
\textbf{Entidades:} & \textbf{CLIENTE} (N) $\longleftrightarrow$ (1) \textbf{REGALO} \\ \hline
\textbf{Descripción:} & Identifica al cliente destinatario de cada regalo.  
Un regalo sólo tiene un receptor, pero un cliente puede recibir muchos regalos. \\ \hline
\textbf{Cardinalidad:} & N (Cliente) : 1 (Regalo) \\ \hline
\end{longtable}

% ==== Relación REALIZA (Cliente – Compra) ====
\begin{longtable}{|p{3.5cm}|p{10cm}|}
\caption{Diccionario de la relación \texttt{realiza} (compras)}
\label{tab:rel_realiza_compra} \\ \hline
\textbf{Nombre:} & realiza \\ \hline
\textbf{Entidades:} & \textbf{CLIENTE} (1) $\longleftrightarrow$ (N) \textbf{COMPRA} \\ \hline
\textbf{Descripción:} & Vincula la compra con el cliente que la efectuó. Un cliente puede realizar varias compras pero una compra le corresponde solo a un cliente. \\ \hline
\textbf{Cardinalidad:} & 1 (Cliente) : N (Compra) \\ \hline
\end{longtable}

% ==== Relación REALIZA (Cliente – Descarga) ====
\begin{longtable}{|p{3.5cm}|p{10cm}|}
\caption{Diccionario de la relación \texttt{realiza} (descargas)}
\label{tab:rel_realiza_descarga} \\ \hline
\textbf{Nombre:} & realiza \\ \hline
\textbf{Entidades:} & \textbf{CLIENTE} (1) $\longleftrightarrow$ (N) \textbf{DESCARGA} \\ \hline
\textbf{Descripción:} & Registra todas las descargas que un cliente ha efectuado. Un cliente puede realizar varias descargas pero una descarga le corresponde solo a un cliente. \\ \hline
\textbf{Cardinalidad:} & 1 (Cliente) : N (Descarga) \\ \hline
\end{longtable}

\newpage
% ==== Relación TIENE (Cliente – Carrito) ====
\begin{longtable}{|p{3.5cm}|p{10cm}|}
\caption{Diccionario de la relación \texttt{tiene} (carrito)}
\label{tab:rel_tiene_carrito} \\ \hline
\textbf{Nombre:} & tiene \\ \hline
\textbf{Entidades:} & \textbf{CLIENTE} (1) $\longleftrightarrow$ (1) \textbf{CARRITO} \\ \hline
\textbf{Descripción:} & Cada cliente posee exactamente un carrito activo, y cada carrito pertenece a un único cliente. \\ \hline
\textbf{Cardinalidad:} & 1 : 1 \\ \hline
\end{longtable}

% ==== Relación RANKEA (Ranking\_Contenido – Contenido) ====
\begin{longtable}{|p{3.5cm}|p{10cm}|}
\caption{Diccionario de la relación \texttt{rankea} (contenido)}
\label{tab:rel_rankea_contenido} \\ \hline
\textbf{Nombre:} & rankea \\ \hline
\textbf{Entidades:} & \textbf{RANKING\_CONTENIDO} (N) $\longleftrightarrow$ (1) \textbf{CONTENIDO} \\ \hline
\textbf{Descripción:} & Posiciona cada contenido dentro de un ranking de popularidad o ventas. \\ \hline
\textbf{Cardinalidad:} & N (Ranking\_Contenido) : 1 (Contenido) \\ \hline
\end{longtable}

% ==== Relación RANKEA (Ranking\_Cliente – Cliente) ====
\begin{longtable}{|p{3.5cm}|p{10cm}|}
\caption{Diccionario de la relación \texttt{rankea} (cliente)}
\label{tab:rel_rankea_cliente} \\ \hline
\textbf{Nombre:} & rankea \\ \hline
\textbf{Entidades:} & \textbf{RANKING\_CLIENTE} (N) $\longleftrightarrow$ (1) \textbf{CLIENTE} \\ \hline
\textbf{Descripción:} & Asocia la posición de un cliente en un ranking (p.ej.\ nivel de actividad). \\ \hline
\textbf{Cardinalidad:} & N (Ranking\_Cliente) : 1 (Cliente) \\ \hline
\end{longtable}