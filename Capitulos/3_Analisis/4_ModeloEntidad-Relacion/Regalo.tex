% ==== Entidad REGALO ====
\renewcommand{\arraystretch}{1.3}
\begin{longtable}{|p{3.5cm}|p{10cm}|}
\caption{Diccionario de la entidad Regalo}
\label{tab:diccionarioRegalo} \\ 
\hline
\textbf{Nombre:} & Regalo \\
\hline
\textbf{Descripción:} & 
Representa la transferencia de un \textbf{Contenido} desde un
\textbf{Cliente} emisor hacia otro cliente receptor como obsequio
dentro de \textbf{QuickContentMedia}.  Guarda la fecha del envío y el
estado (\textit{abierto}/\textit{pendiente}). \\ \hline
\endfirsthead

\multicolumn{2}{c}{\textbf{Continuación desde la página anterior}} \\ 
\endhead

\hline \multicolumn{2}{r}{{Continúa en la siguiente página}} \\ 
\endfoot

\hline
\endlastfoot

\multicolumn{2}{|p{13.5cm}|}{\textbf{ATRIBUTOS}} \\ \hline
\textbf{Atributo} & \textbf{Descripción} \\ \hline
fecha        & Fecha en la que el emisor envía el contenido. \\ \hline
abierto      & Bandera lógica que indica si el receptor ya abrió el regalo
(1 = abierto, 0 = pendiente). \\ \hline

\multicolumn{2}{|p{13.5cm}|}{\textbf{RELACIONES}} \\ \hline
\textbf{Relación} & \textbf{Descripción} \\ \hline
emisor\,(N\,:\,1)   & Un cliente \textbf{(emisor)} puede enviar muchos regalos. \\ \hline
receptor\,(N\,:\,1) & Un cliente \textbf{(receptor)} puede recibir muchos regalos. \\ \hline
contiene\,(N\,:\,1) & Cada regalo \emph{contiene} exactamente un \textbf{Contenido}.\\ \hline
\end{longtable}