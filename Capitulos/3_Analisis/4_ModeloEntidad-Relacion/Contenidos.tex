% ==== Entidad CONTENIDO ====
\renewcommand{\arraystretch}{1.3}
\begin{longtable}{|p{3.5cm}|p{10cm}|}
\caption{Diccionario de la entidad Contenido}
\label{tab:diccionarioContenido} \\ 
\hline
\textbf{Nombre:} & Contenido \\
\hline
\textbf{Descripción:} & 
Objeto multimedia disponible en \textbf{QuickContentMedia}. Puede ser comprado,
descargado, regalado y calificado por los clientes. \\ \hline
\endfirsthead

\multicolumn{2}{c}{\textbf{Continuación desde la página anterior}} \\ 
\endhead

\hline \multicolumn{2}{r}{{Continúa en la siguiente página}} \\ 
\endfoot

\hline
\endlastfoot

\multicolumn{2}{|p{13.5cm}|}{\textbf{ATRIBUTOS}} \\ \hline
\textbf{Atributo} & \textbf{Descripción} \\ \hline
formato         & Tipo genérico (\textit{video}, \textit{imagen}, \textit{audio}). \\ \hline
autor           & Nombre del creador o propietario intelectual. \\ \hline
archivo         & Ruta o nombre del archivo almacenado. \\ \hline
nombre          & Título descriptivo que se muestra al usuario. \\ \hline
precio          & Precio de venta vigente en la plataforma. \\ \hline
tamano\_archivo & Peso del archivo en MB. \\ \hline
descripcion     & Sinopsis o descripción breve del contenido. \\ \hline

\multicolumn{2}{|p{13.5cm}|}{\textbf{RELACIONES}} \\ \hline
\textbf{Relación} & \textbf{Descripción} \\ \hline
tiene\,(N\,:\,1) & Cada \textbf{Contenido} \textit{tiene} un único \textbf{Tipo\_Archivo}; un tipo de archivo puede describir muchos contenidos. \\ \hline
pertenece\,(N\,:\,1) & Un contenido \textit{pertenece} a una \textbf{Categoria}; una categoría agrupa muchos contenidos. \\ \hline
aplica\,(N\,:\,1) & Opcionalmente, una \textbf{Promocion} \textit{aplica} a múltiples contenidos; un contenido solo puede tener una promoción activa. \\ \hline
contiene(Descarga)\,(1\,:\,N) & Un contenido puede estar \textit{contenido} en muchas \textbf{Descargas}; cada descarga hace referencia a un único contenido. \\ \hline
contiene(Regalo)\,(1\,:\,N) & Un \textbf{Regalo} \textit{contiene} exactamente un contenido; un mismo contenido puede ser regalado múltiples veces. \\ \hline
\end{longtable}