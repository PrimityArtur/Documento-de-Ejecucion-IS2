% ==== Entidad CALIFICACION ====
\renewcommand{\arraystretch}{1.3}
\begin{longtable}{|p{3.5cm}|p{10cm}|}
\caption{Diccionario de la entidad Calificación}
\label{tab:diccionarioCalificacion} \\ 
\hline
\textbf{Nombre:} & Calificación \\
\hline
\textbf{Descripción:} & 
Registro de la valoración (1 – 10) que un \textbf{Cliente} otorga a un
\textbf{Contenido} descargado. \\ \hline
\endfirsthead

\multicolumn{2}{c}{\textbf{Continuación desde la página anterior}} \\ 
\endhead

\hline \multicolumn{2}{r}{{Continúa en la siguiente página}} \\ 
\endfoot

\hline
\endlastfoot

\multicolumn{2}{|p{13.5cm}|}{\textbf{ATRIBUTOS}} \\ \hline
\textbf{Atributo} & \textbf{Descripción} \\ \hline
fecha            & Fecha en la que el cliente emitió la valoración. \\ \hline
nota             & Puntuación asignada al contenido (entero 1–10). Puede ser nula mientras el cliente aún no califique. \\ \hline

\multicolumn{2}{|p{13.5cm}|}{\textbf{RELACIONES}} \\ \hline
\textbf{Relación} & \textbf{Descripción} \\ \hline
realiza\,(N\,:\,1) & Un \textbf{Cliente} \textit{realiza} muchas calificaciones;  
cada calificación pertenece a un único cliente. \\ \hline
está\,(N\,:\,1) & Varias calificaciones \textit{están} asociadas a un  
\textbf{Ranking} semanal; cada calificación puede ubicarse en un único ranking. \\ \hline
\end{longtable}