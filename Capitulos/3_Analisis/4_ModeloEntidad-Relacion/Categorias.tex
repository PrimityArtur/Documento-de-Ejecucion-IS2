% ==== Entidad CATEGORIA ====
\renewcommand{\arraystretch}{1.3}
\begin{longtable}{|p{3.5cm}|p{10cm}|}
\caption{Diccionario de la entidad Categoría}
\label{tab:diccionarioCategoria} \\ 
\hline
\textbf{Nombre:} & Categoría \\
\hline
\textbf{Descripción:} & 
Estructura jerárquica que clasifica los contenidos en
\textbf{QuickContentMedia}.  Cada categoría puede funcionar como raíz o
sub-categoría. \\ \hline
\endfirsthead

\multicolumn{2}{c}{\textbf{Continuación desde la página anterior}} \\ 
\endhead

\hline \multicolumn{2}{r}{{Continúa en la siguiente página}} \\ 
\endfoot

\hline
\endlastfoot

\multicolumn{2}{|p{13.5cm}|}{\textbf{ATRIBUTOS}} \\ \hline
\textbf{Atributo} & \textbf{Descripción} \\ \hline
nombre             & Título breve de la categoría. \\ \hline

\multicolumn{2}{|p{13.5cm}|}{\textbf{RELACIONES}} \\ \hline
\textbf{Relación} & \textbf{Descripción} \\ \hline
pertenece\,(N\,:\,1) & Una sub-categoría \textit{pertenece} a
                       una categoría padre.  Cardinalidad N:1 (muchas hijas – un padre). \\ \hline
tiene\,(1\,:\,N) & Una categoría \textit{tiene} muchos
                   \textbf{Contenidos}; cada contenido pertenece a una sola categoría. \\ \hline
\end{longtable}