% ==== Entidad RANKING ====
\renewcommand{\arraystretch}{1.3}
\begin{longtable}{|p{3.5cm}|p{10cm}|}
\caption{Diccionario de la entidad Ranking}
\label{tab:diccionarioRanking} \\ 
\hline
\textbf{Nombre:} & Ranking \\
\hline
\textbf{Descripción:} & 
Tabla que consolida la posición de \textbf{Contenidos} o
\textbf{Clientes} de acuerdo con un criterio (\textit{tipo}) como número
de descargas o nota promedio.  Cada registro representa un ranking
vigente. \\ \hline
\endfirsthead

\multicolumn{2}{c}{\textbf{Continuación desde la página anterior}} \\ 
\endhead

\hline \multicolumn{2}{r}{{Continúa en la siguiente página}} \\ 
\endfoot

\hline
\endlastfoot

\multicolumn{2}{|p{13.5cm}|}{\textbf{ATRIBUTOS}} \\ \hline
\textbf{Atributo} & \textbf{Descripción} \\ \hline
tipo                   & Tipo de clasificación generada  
(\emph{descargas}, \emph{valoración}, \emph{clientes}). \\ \hline
fecha\_inicio\_semanal & Fecha que marca el comienzo de la semana sobre la cual se calcula el ranking. \\ \hline

\multicolumn{2}{|p{13.5cm}|}{\textbf{RELACIONES}} \\ \hline
\textbf{Relación} & \textbf{Descripción} \\ \hline
está(descarga)\,(1\,:\,N) & Un \textbf{Ranking} \textit{está} asociado a muchas \textbf{Descargas};
cada descarga pertenece a un único ranking semanal. \\ \hline
está(calificación)\,(1\,:\,N) & Un ranking \textit{está} asociado a muchas \textbf{Calificaciones};
cada calificación pertenece a un único ranking semanal. \\ \hline
\end{longtable}