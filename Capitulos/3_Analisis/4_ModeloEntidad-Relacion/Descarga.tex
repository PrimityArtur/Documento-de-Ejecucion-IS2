% ==== Entidad DESCARGA ====
\renewcommand{\arraystretch}{1.3}
\begin{longtable}{|p{3.5cm}|p{10cm}|}
\caption{Diccionario de la entidad Descarga}
\label{tab:diccionarioDescarga} \\ 
\hline
\textbf{Nombre:} & Descarga \\
\hline
\textbf{Descripción:} & 
Registra cada ocasión en la que un \textbf{Cliente} baja un
\textbf{Contenido}. \\ \hline
\endfirsthead

\multicolumn{2}{c}{\textbf{Continuación desde la página anterior}} \\ 
\endhead

\hline \multicolumn{2}{r}{{Continúa en la siguiente página}} \\ 
\endfoot

\hline
\endlastfoot

\multicolumn{2}{|p{13.5cm}|}{\textbf{ATRIBUTOS}} \\ \hline
\textbf{Atributo} & \textbf{Descripción} \\ \hline
fecha        & Fecha–hora en que el usuario realizó la descarga. \\ \hline

\multicolumn{2}{|p{13.5cm}|}{\textbf{RELACIONES}} \\ \hline
\textbf{Relación} & \textbf{Descripción} \\ \hline
contiene\,(1\,:\,N) & Un \textbf{Contenido} \textit{contiene} muchas descargas;  
cada descarga referencia exactamente un contenido. \\ \hline
realiza\,(N\,:\,1) & Un \textbf{Cliente} \textit{realiza} muchas descargas;  
cada descarga pertenece a un único cliente. \\ \hline
está\,(N\,:\,1) & Varias descargas \textit{están} asociadas a un  
\textbf{Ranking} semanal; un ranking agrupa muchas descargas. \\ \hline
\end{longtable}