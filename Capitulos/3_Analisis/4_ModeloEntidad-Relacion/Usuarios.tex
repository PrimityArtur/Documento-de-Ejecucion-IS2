% ==== Entidad USUARIO ====
\renewcommand{\arraystretch}{1.3}
\begin{longtable}{|p{3.5cm}|p{10cm}|}
\caption{Diccionario de la entidad Usuario}
\label{tab:diccionarioUsuario} \\ 
\hline
\textbf{Nombre:} & Usuario \\ \hline
\textbf{Descripción:} & 
Entidad genérica que agrupa a todos los usuarios del sistema
\textbf{QuickContentMedia}.  
De ella se derivan dos sub-tipos: \textit{Cliente} (posee saldo,
descarga contenidos, realiza compras) y \textit{Administrador}
(ejecuta tareas de gestión). \\ \hline
\endfirsthead

\multicolumn{2}{c}{\textbf{Continuación desde la página anterior}} \\ 
\endhead

\hline \multicolumn{2}{r}{{Continúa en la siguiente página}} \\ 
\endfoot

\hline
\endlastfoot

\multicolumn{2}{|p{13.5cm}|}{\textbf{ATRIBUTOS}} \\ \hline
\textbf{Atributo} & \textbf{Descripción} \\ \hline
username    & Nombre de usuario utilizado para autenticarse.  
Valor único en la base de datos. \\ \hline
contrasena  & Clave de acceso cifrada. \\ \hline
nombre      & Nombre de pila del usuario (cadena de hasta 20 caracteres). \\ \hline
apellido    & Apellidos del usuario (cadena de hasta 255 caracteres). \\ \hline

\multicolumn{2}{|p{13.5cm}|}{\textbf{RELACIONES}} \\ \hline
\textbf{Relación} & \textbf{Descripción} \\ \hline
\textit{Especialización} &  
Un \textbf{Usuario} se especializa en:  
\begin{itemize}
  \item \textbf{Cliente} – accede a funcionalidades de compra, descarga, ranking, etc.  
  \item \textbf{Administrador} – gestiona contenidos, categorías, clientes y promociones.
\end{itemize} \\ \hline
\end{longtable}