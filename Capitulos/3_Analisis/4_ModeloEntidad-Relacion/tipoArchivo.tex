% ==== Entidad TIPO_ARCHIVO ====
\renewcommand{\arraystretch}{1.3}
\begin{longtable}{|p{3.5cm}|p{10cm}|}
\caption{Diccionario de la entidad Tipo\_Archivo}
\label{tab:diccionarioTipoArchivo} \\ 
\hline
\textbf{Nombre:} & Tipo\_Archivo \\
\hline
\textbf{Descripción:} & 
Catálogo de extensiones y metadatos de archivo admitidos por
\textbf{QuickContentMedia}. \\ \hline
\endfirsthead

\multicolumn{2}{c}{\textbf{Continuación desde la página anterior}} \\ 
\endhead

\hline \multicolumn{2}{r}{{Continúa en la siguiente página}} \\ 
\endfoot

\hline
\endlastfoot

\multicolumn{2}{|p{13.5cm}|}{\textbf{ATRIBUTOS}} \\ \hline
\textbf{Atributo} & \textbf{Descripción} \\ \hline
extension         & Extensión típica del archivo (p.~ej.\ \texttt{.mp4}, \texttt{.jpg}). \\ \hline
tipo\_contenido   & Categoría del archivo (\textit{video}, \textit{imagen}, \textit{audio}, etc.). \\ \hline
mime\_type        & Cadena MIME completa (p.~ej.\ \texttt{video/mp4}, \texttt{image/jpeg}). \\ \hline

\multicolumn{2}{|p{13.5cm}|}{\textbf{RELACIONES}} \\ \hline
\textbf{Relación} & \textbf{Descripción} \\ \hline
tiene\,(1\,:\,N) & Un \textbf{Tipo\_Archivo} \textit{tiene} asociados muchos \textbf{Contenidos}; cada contenido referencia exactamente un tipo de archivo. \\ \hline
\end{longtable}