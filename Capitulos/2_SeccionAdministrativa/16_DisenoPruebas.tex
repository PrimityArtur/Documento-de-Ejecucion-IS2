\subsection{Diseño de casos de prueba}
Los diseños de casos de prueba se codifican con la letra P seguida de una letra que identifica el tipo de prueba (U para unitaria, I para integración, V para validación y S para sistema) y un número único secuencial. 

%%%%%%%%%%%%%%%%%%% TABLA %%%%%%%%%%%%%%%%%%% 
\renewcommand{\arraystretch}{1.3} % para mejor espaciado vertical
\begin{longtable}{|c|p{10cm}|}
\caption{Codificación de Diseños de casos de prueba} \\
\hline
\textbf{Código} & \textbf{Nombre} \\
\hline
\endfirsthead

\multicolumn{2}{c}%
{{\bfseries \tablename\ \thetable{} -- continuación desde la página anterior}} \\
\hline
\textbf{Código} & \textbf{Diseño de caso de prueba} \\
\hline
\endhead

\hline \multicolumn{2}{r}{{Continúa en la siguiente página}} \\
\endfoot

\hline
\endlastfoot

%%%%%%%%%%%%%%%%%%%v
PU-001 & (Unitaria) Verificación de agregar contenido al carrito de compras \\
\hline
PU-002 & (Unitaria) Verificación de actualización de saldo de cliente \\
\hline
PI-001 & (Integración) Integración durante una compra con descuento \\
\hline
PI-002 & (Integración) Integración al visualizar el historial de descargas de un cliente \\
\hline
PV-001 & (Validación) Validación del ingreso de saldo a un cliente desde el panel del
administrador \\
\hline
PV-002 & (Validación) Validación del ingreso y verificación del usuario destinatario
antes del envío de regalo \\
\hline
PS-001 & (Sistema) Validación de compatibilidad de interfaz y funcionalidad en
múltiples navegadores \\
\hline
PS-002 & (Sistema) Validación de restricciones de acceso a funciones administrati-
vas desde cuentas no autorizadas \\
\hline
\end{longtable}