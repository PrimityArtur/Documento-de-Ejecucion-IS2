\subsection{Casos de Uso}
Los casos de uso son codificados con con el nombre CU y se les asigna un número único y secuencial.

%%%%%%%%%%%%%%%%%%% TABLA %%%%%%%%%%%%%%%%%%% 
\renewcommand{\arraystretch}{1.3} % para mejor espaciado vertical
\begin{longtable}{|c|c|}
\caption{Codificación de Casos de Uso} \\
\hline
\textbf{Código} & \textbf{Caso de Uso} \\
\hline
\endfirsthead

\multicolumn{2}{c}%
{{\bfseries \tablename\ \thetable{} -- continuación desde la página anterior}} \\
\hline
\textbf{Código} & \textbf{Caso de Uso} \\
\hline
\endhead

\hline \multicolumn{2}{r}{{Continúa en la siguiente página}} \\
\endfoot

\hline
\endlastfoot

%%%%%%%%%%%%%%%%%%%v
CU-001 & Acceder al portal \\
\hline

%%%%%%%%%%%%%%%%%%%
CU-002 & Administrar cuenta \\
\hline

%%%%%%%%%%%%%%%%%%%
CU-003 & Agregar saldo \\
\hline

%%%%%%%%%%%%%%%%%%%
CU-004 & Navegar y seleccionar contenido \\
\hline

%%%%%%%%%%%%%%%%%%%
CU-005 & Administrar carrito de compras \\
\hline

%%%%%%%%%%%%%%%%%%%
CU-006 & Administrar contenido adquirido \\
\hline

%%%%%%%%%%%%%%%%%%%
CU-007 & Administrar promociones \\
\hline

%%%%%%%%%%%%%%%%%%%
CU-008 & Administrar clientes \\
\hline

%%%%%%%%%%%%%%%%%%%
CU-009 & Administrar contenidos \\
\hline

%%%%%%%%%%%%%%%%%%%
CU-010 & Administrar categorías \\
\hline

%%%%%%%%%%%%%%%%%%%
\end{longtable}
