%--------------------------------------------------------------
\subsection{Copias de seguridad}

Para realizar las copias de seguridad de la base de datos del sistema \textbf{QuickContentMedia} se realizarán las siguuientes acciones:

\begin{enumerate}
  \item \textbf{Copia completa semanal} (domingo 02:00)  
        \texttt{pg\_dumpall --format=custom}  
  \item \textbf{Copia semanal diferencial de archivos multimedia} (rsync)  
        (miércoles 02:00) en el directorio \texttt{/backups/media}.
  \item \textbf{Sincronización en la nube} una vez por semana, tras la
        copia completa, con \texttt{rclone} a un bucket S3.  
  \item \textbf{Retención}  
        \begin{itemize}
          \item Últimas 4 copias completas locales: 4 semanas.  
          \item Copias en nube: 3 meses.  
        \end{itemize}
\end{enumerate}
La tabla \ref{tab:bkp_small} resume el calendario de los backups a realizarse.
\begin{table}[htbp]
\centering
\caption{Calendario resumido de backups}\label{tab:bkp_small}
\begin{tabular}{|l|c|c|l|}
\hline
\textbf{Componente} & \textbf{Tipo} & \textbf{Frecuencia} & \textbf{Herramienta} \\ \hline
Base de datos & Completa & Semanal (dom) & pg\_dumpall –Fc \\ \hline
Multimedia    & Diferencial & Semanal (mié) & rsync --link-dest \\ \hline
Off-site      & Sincronización & Semanal (dom) & rclone a S3 \\ \hline
\end{tabular}
\end{table}
