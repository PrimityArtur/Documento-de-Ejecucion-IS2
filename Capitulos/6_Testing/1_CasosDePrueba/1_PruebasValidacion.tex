\subsection{Casos de prueba de validación}
A continuación se presentan los casos de prueba de validación construidos.  

\begin{table}[H]
\centering
\caption{Caso de prueba \textbf{CPV-001}}
\label{tab:cpv001Saldo}
\begin{tabular}{|p{4cm}|p{10.5cm}|}
\hline
\textbf{Campo} & \textbf{Descripción} \\ \hline
\textbf{Código} & CPV-001 \\ \hline
\textbf{Diseño de prueba} & PV-001 \\ \hline
\textbf{Valor a ingresar} & 150 (entero no negativo) \\ \hline
\textbf{Ejecutores} & Camila Salazar – Tester junior \\ \hline
\textbf{Usuario en la prueba} & Guillermo Calderón \\ \hline
\textbf{Pantalla inicial} & UIAdministrar-
Cliente (MK-038) \\ \hline
\textbf{Pasos ejecutados} &
\begin{enumerate}
  \item Clic en \textit{Gestionar saldo} del cliente \#02.
  \item En la ventana emergente introduce el valor \textbf{150}.
  \item Clic en \textit{Confirmar}.
\end{enumerate} \\ \hline
\textbf{Resultado esperado} &
\begin{itemize}
  \item El sistema acepta el valor sin mostrar errores.
  \item El saldo del cliente \#02 se actualiza a “150” en la tabla de la interfaz.
\end{itemize} \\ \hline
\textbf{Resultado obtenido} &  \\ \hline
\textbf{Estado} &  \\ \hline
\end{tabular}
\end{table}

\begin{table}[H]
\centering
\caption{Caso de prueba \textbf{CPV-002}}
\label{tab:cpv003AgregarContenido}
\begin{tabular}{|p{4cm}|p{10.5cm}|}
\hline
\textbf{Campo} & \textbf{Descripción} \\ \hline
\textbf{Código} & CPV-002 \\ \hline
\textbf{Diseño de prueba} & PV-002 \\ \hline
\textbf{Valores a ingresar} &
\begin{itemize}
  \item Tipo de contenido: \textbf{Imagen}
  \item Nombre: \texttt{Chrome}
  \item Descripción: \texttt{Google}
  \item Categoría: \texttt{Software}
  \item Autor: \texttt{Pedro}
  \item Precio: \texttt{15}
  \item Archivo: \texttt{google\_chrome.jpg}
\end{itemize} \\ \hline
\textbf{Ejecutores} & Camila Salazar – Tester junior \\ \hline
\textbf{Usuario en la prueba} & Guillermo Calderón – Administrador \\ \hline
\textbf{Pantalla inicial} & \textbf{UIAdministrarContenido (MK-030)} \\ \hline
\textbf{Pasos ejecutados} &
\begin{enumerate}
  \item Clic en \textit{Agregar contenido}.
  \item En la ventana emergente seleccionar \textbf{Imagen}.
  \item Completar los campos con los valores listados arriba.
  \item Pulsar el icono de subir archivo, elegir \texttt{google\_chrome.jpg} y aceptar.
  \item Clic en el botón \textit{Agregar}.
\end{enumerate} \\ \hline
\textbf{Resultado esperado} &
\begin{itemize}
  \item El sistema acepta los datos sin error.
  \item El nuevo registro \texttt{Chrome} se muestra en la tabla de contenidos con tipo “Imagen”.
\end{itemize} \\ \hline
\textbf{Resultado obtenido} &  \\ \hline
\textbf{Estado} &  \\ \hline
\end{tabular}
\end{table}